%-------------------------------------------------------------------------------
%	SECTION TITLE
%-------------------------------------------------------------------------------
\cvsection{Education}
%-------------------------------------------------------------------------------
%	CONTENT
%-------------------------------------------------------------------------------
\begin{cventries}

%---------------------------------------------------------
  \cventry
    {PhD in the School of Aerospace Engineering} % Degree
    {University of Liverpool} % Institution
    {Liverpool, UK} % Location
    {March 2024 - Current} % Date(s)
    {
      \begin{cvitems} % Description(s) bullet points
        \item {Geophysics PhD. Co-Funded by the School of Engineering, and the European Space Agency.}
        \item {Primary supervisor at the University of Liverpool is \href{https://www.liverpool.ac.uk/engineering/staff/stefania-soldini/}{\color{darkblue}{Dr Stefania Soldini}} 
        }
        \item {Potential (unconfirmed) for up to 6 months to be spent at \href{https://maps.app.goo.gl/FiTspiLjZnPt5VcE6}{\color{darkblue}{ESOC}} (European Space Operations Centre), to contribute to the development of \href{https://godot.io.esa.int/docs/1.4.0/}{\color{darkblue}{GODOT}}.
        }
      \end{cvitems}
    }
    
  \cventry
    {BSc (Hons) Geophysics (Physics)} % Degree
    {University of Liverpool} % Institution
    {Liverpool, UK} % Location
    {September 2020 - July 2023} % Date(s)
    {
      \begin{cvitems} % Description(s) bullet points
        \item {Graduated with a first class honours degree, accredited by the \href{https://www.iop.org/}{\color{darkblue}{Institute of Physics}}}
        \item {Dissertation titled: "The magnetic fields of Uranus and Neptune from Voyager data", supervised by \href{https://www.liverpool.ac.uk/environmental-sciences/staff/richard-holme/}{\color{darkblue}{Professor Richard Holme}}
        }
        \item {Introduced principles such as: remote sensing using Python with data acquired by the LANDSAT 8 Satellite, Machine learning and non-linear inverse methods with gravity applications, tools on how to produce Geophysical Models in Linux, the dangers of non-uniqueness,  matrix analysis, and optimisation theory and statistics. }
        \item {There were fieldwork elements that ran with the input of \href{https://www.sepgeophysical.com/}{\color{darkblue}{SEP Geophysical}}, where I learned how to use equipment and
            techniques used in Exploration Geophysics such as: EM31/34, Seismic Refraction, Magnetic Gradiometry,
            and Ground Penetrating Radar}
      \end{cvitems}
    }
    
  \cventry
    {3 A-Levels} %Courses 
    {Aquinas College} % Institution
    {Stockport, UK} % Location
    {September 2018 - June 2020} % Date(s)
    {
      \begin{cvitems} % Description(s) bullet points
        \item {Geography, Maths and Physics}
        \item {Centre assessed at grade B (Due to COVID-19 pandemic there was no formal assessment)}
      \end{cvitems}
    }

    \cventry
        {12 GCSEs or Equivalent BTEC} %Courses
        {St Thomas More School, and Royal Air Force Air Cadets} % Institution
        {Buxton, UK} %Location
        {September 2013 - June 2018}
    {
      \begin{cvitems} % Description(s) bullet points
        \item {BTECs include: Teamwork and personal development in the community, Aviation Studies, and PE (Physical Education)}
        \item {All subjects passed, including all required fundamentals including English, Maths and Science}
      \end{cvitems}
    }
        
%---------------------------------------------------------
\end{cventries}
